\documentclass[12pt]{article}
\pagestyle{empty}
\usepackage[utf8]{inputenc}
\usepackage[russian]{babel}
\usepackage[top=1cm,left=1cm,right=2cm, bottom=1cm]{geometry}
\usepackage{tabularx}
\begin{document}


\begin{center}
\quad Университет ИТМО, факультет программной инженерии и компьютерной техники \\
\quad Двухнедельная отчётная работа по «Информатике»: аннотация к статье\\
\quad Дата прошедшей лекции: \underline{\hspace{2.3cm}} 	Номер прошедшей лекции: \underline{\hspace{0.8cm}}	Дата сдачи: \underline{\hspace{2.3cm}}

\bigskip

\quad Выполнил(а) \underline{\hspace{5cm}}, № группы \underline{ P31XX }, оценка \underline{\hspace{2cm}}


\end{center}

\begin{tabularx}{\textwidth} { 
  | >{\raggedright\arraybackslash}X|}
    \hline
\textbf{Название статьи/главы книги/видеолекции}\\
    \\
    \hline
\end{tabularx}

\begin{tabularx}{\textwidth} 
{ 
| >{\centering\arraybackslash}X
| >{\centering\arraybackslash}X
| >{\centering\arraybackslash}X 
|}
    \textbf{ФИО автора статьи \quad (или e-mail)} & \textbf{Дата публикации \qquad\qquad (не старше 2020 года)} & \textbf{Размер статьи \qquad\qquad (от 400 слов)} \\
     \textit{Для Хабра достаточно указать ник автора. Для сайтов крупных IT-компаний указывать автора не обязательно (ibm.com, microsoft.com, intel.com).} & <<\underline{\hspace{0.5cm}}>> \underline{\hspace{1.5cm}} 202\underline{\hspace{0.2cm}} г. & \underline{\hspace{2.5cm}} \\
    \hline
\end{tabularx}

\begin{tabularx}{\textwidth} { 
  | >{\raggedright\arraybackslash}X|}
    \textbf{Прямая полная ссылка на источник или сокращённая ссылка (bit.ly, tr.im и т.п.)} \\
    \\
    \smallskip\\
    \hline
    \textbf{Теги, ключевые слова или словосочетания}\\
    \\
    \smallskip\\
    \hline
    \textbf{Перечень фактов, упомянутых в статье (минимум три пункта)}\\
    ...\\
    n)\\
    \textit{Каждый из n пунктов должен представлять из себя ровно одно предложение, написанное своими словами (прямые цитаты из исходного документа недопустимы). Предложения не должны быть связанны друг с другом грамматически. Наличие грамматических и пунктуационных ошибок не влияет на оценку. Допускается привести только такое количество фактов, чтобы вся аннотация умещалась на одну страницу А4. Порядок перечисления должен совпадать с порядком описания фактов в оригинальном источнике.}
    \smallskip\\
    \textit{При выборе статьи следует ориентироваться на «критерий Пушкина»: статья должна содержать такое количество технической информации, чтобы человек с гуманитарным образованием мало чего бы понял при чтении (описание алгоритмов, формулы, концепции языков программирования, физические принципы, IT-технологии и т.п.).}
    \hline
    \textbf{Позитивные следствия и/или достоинства описанной в статье технологии (минимум три пункта)}\\
    1) \\
    2) \\
    3) \\
    \hline
    \textbf{Негативные следствия и/или достоинства описанной в статье технологии (минимум три пункта)}\\
    1) \\
    2) \\
    3) \\
    \hline
    \textbf{Ваши замечания, пожелания преподавателю или анекдот о программистах}\footnote{Наличие этой графы не влияет на оценку}\\
    \bigskip\\
    \bigskip\\
    \hline
    
\end{tabularx}


\end{document}
