\documentclass[a4paper,12pt]{article}

% ---------------------------------------------------------------------------
\usepackage[T2A]{fontenc}
\usepackage[utf8]{inputenc}
\usepackage[russian]{babel}
\usepackage{array,longtable,booktabs,amsmath}
\usepackage{geometry}
\usepackage{hyperref}
\usepackage{listings}          % <‑‑ форматирование кода
\usepackage{xcolor}
\geometry{margin=2.5cm}

\newcommand{\code}[1]{\texttt{#1}}

% ----------- настройки listings ---------------------------------------------
\lstset{
  basicstyle=\small\ttfamily,
  breaklines=true,
  frame=single,
  columns=fullflexible,
  keywordstyle=\color{blue}\bfseries,
  commentstyle=\color{gray},
  stringstyle=\color{teal},
  showstringspaces=false,
  language=Python
}

% ---------------------------------------------------------------------------
\begin{document}

\begin{center}
  \Large\bfseries
  ЛР6 Чураков А А P3231 В-19.\\
  Генетический алгоритм.\\
  Задача коммивояжёра.
\end{center}

\vspace{0.5em}\hrule\vspace{1em}

\section*{1. Исходные данные}

\begin{itemize}
  \item Число городов: $5$.
  \item Размер популяции: $N_{\text{pop}} = 4$.
  \item Вероятность мутации (обмен двух генов): $p_{\text{mut}} = 0{,}01$.
  \item Матрица расстояний $D$:\\[0.3em]
\end{itemize}

\begin{center}
\begin{tabular}{c|ccccc}
\toprule
&0&1&2&3&4\\\midrule
0&0&4&6&2&9\\
1&4&0&3&2&9\\
2&6&3&0&5&9\\
3&2&2&5&0&8\\
4&9&9&9&8&0\\
\bottomrule
\end{tabular}
\end{center}

% =========================================================================== %
\section*{2. Поколение 1}

\subsection*{2.1 Начальная популяция}

\begin{center}
\begin{tabular}{ccl}
\toprule
№ & Маршрут & Длина\\\midrule
1 & 4\,2\,1\,0\,3 & 26\\
2 & 4\,2\,0\,3\,1 & 28\\
3 & 4\,2\,3\,1\,0 & 29\\
4 & 4\,2\,0\,1\,3 & 29\\
\bottomrule
\end{tabular}
\end{center}

\subsection*{2.2 Скрещивания и мутации}

\paragraph{Пара 1 (индексы $[0,3]$, точки $1,2$).}
\begin{longtable}{@{}p{0.18\linewidth}p{0.75\linewidth}@{}}
Родитель 1 & 4 \,|\, 2\,1 \,|\, 0\,3\\
Родитель 2 & 4 \,|\, 2\,0 \,|\, 1\,3\\
Потомок 1  & 1\,2\,0\,3\,4\\
Потомок 2  & 0\,2\,1\,3\,4\\
Мутация    & не была выполнена ($p_{\text{mut}}=0{,}01$).\\
\end{longtable}

\paragraph{Пара 2 (индексы $[0,1]$, точки $1,2$).}
\begin{longtable}{@{}p{0.18\linewidth}p{0.75\linewidth}@{}}
Родитель 1 & 4 \,|\, 2\,1 \,|\, 0\,3\\
Родитель 2 & 4 \,|\, 2\,0 \,|\, 3\,1\\
Потомок 1  & 1\,2\,0\,3\,4\\
Потомок 2  & 0\,2\,1\,3\,4\\
Мутация    & не была выполнена.\\
\end{longtable}

\subsection*{2.3 Расширенная популяция, отбор лучших 4}

\begin{center}
\begin{tabular}{>{\ttfamily}ccl}
\toprule
ID & Маршрут & Длина\\\midrule
P\_1 & 4\,2\,1\,0\,3 & 26\\
P\_2 & 4\,2\,0\,1\,3 & 29\\
P\_3 & 4\,2\,1\,0\,3 & 26\\
P\_4 & 4\,2\,0\,3\,1 & 28\\ \midrule
C\_1 & 1\,2\,0\,3\,4 & 28\\
C\_2 & 0\,2\,1\,3\,4 & 28\\
C\_3 & 1\,2\,0\,3\,4 & 28\\
C\_4 & 0\,2\,1\,3\,4 & 28\\
\bottomrule
\end{tabular}
\end{center}

\begin{center}
\begin{tabular}{ccl}
\toprule
№ & Маршрут & Длина\\\midrule
1 & 4\,2\,1\,0\,3 & 26\\
2 & 4\,2\,0\,3\,1 & 28\\
3 & 1\,2\,0\,3\,4 & 28\\
4 & 0\,2\,1\,3\,4 & 28\\
\bottomrule
\end{tabular}
\end{center}

% =========================================================================== %
\section*{3. Поколение 2}

\subsection*{3.1 Текущая популяция}

Та же, что после отбора в конце поколения 1.
\begin{center}
\begin{tabular}{ccl}
\toprule
№ & Маршрут & Длина\\\midrule
1 & 4\,2\,1\,0\,3 & 26\\
2 & 4\,2\,0\,3\,1 & 28\\
3 & 1\,2\,0\,3\,4 & 28\\
4 & 0\,2\,1\,3\,4 & 28\\
\bottomrule
\end{tabular}
\end{center}

\subsection*{3.2 Скрещивания и мутации}

\paragraph{Пара 1 (индексы $[1,0]$, точки $1,3$).}
\begin{longtable}{@{}p{0.18\linewidth}p{0.75\linewidth}@{}}
Родитель 1 & 4 \,|\, 2\,0\,3 \,|\, 1\\
Родитель 2 & 4 \,|\, 2\,1\,0 \,|\, 3\\
Потомок 1  & 3\,2\,1\,0\,4\\
Потомок 2  & 1\,2\,0\,3\,4\\
Мутация    & не выполнена.\\
\end{longtable}

\paragraph{Пара 2 (индексы $[2,3]$, точки $1,2$).}
\begin{longtable}{@{}p{0.18\linewidth}p{0.75\linewidth}@{}}
Родитель 1 & 1 \,|\, 2\,0 \,|\, 3\,4\\
Родитель 2 & 0 \,|\, 2\,1 \,|\, 3\,4\\
Потомок 1  & 0\,2\,1\,3\,4\\
Потомок 2  & 1\,2\,0\,3\,4\\
Мутация    & не выполнена.\\
\end{longtable}

\begin{center}
\begin{tabular}{>{\ttfamily}ccl}
\toprule
ID & Маршрут & Длина\\\midrule
P\_1 & 4\,2\,0\,3\,1 & 28\\
P\_2 & 4\,2\,1\,0\,3 & 26\\
P\_3 & 1\,2\,0\,3\,4 & 28\\
P\_4 & 0\,2\,1\,3\,4 & 28\\ \midrule
C\_5 & 3\,2\,1\,0\,4 & 29\\
C\_6 & 1\,2\,0\,3\,4 & 28\\
C\_7 & 0\,2\,1\,3\,4 & 28\\
C\_8 & 1\,2\,0\,3\,4 & 28\\
\bottomrule
\end{tabular}
\end{center}

\subsection*{3.3 Популяция после отбора}

\begin{center}
\begin{tabular}{ccl}
\toprule
№ & Маршрут & Длина\\\midrule
1 & 4\,2\,1\,0\,3 & 26\\
2 & 4\,2\,0\,3\,1 & 28\\
3 & 1\,2\,0\,3\,4 & 28\\
4 & 0\,2\,1\,3\,4 & 28\\
\bottomrule
\end{tabular}
\end{center}

% =========================================================================== %
\section*{4. Поколение 3}

\subsection*{4.1 Текущая популяция}

Идентична популяции конца поколения 2.

\subsection*{4.2 Скрещивания и мутации}

\paragraph{Пара 1 (индексы $[0,2]$, точки $2,3$).}
\begin{longtable}{@{}p{0.18\linewidth}p{0.75\linewidth}@{}}
Родитель 1 & 4\,2 \,|\, 1\,0 \,|\, 3\\
Родитель 2 & 1\,2 \,|\, 0\,3 \,|\, 4\\
Потомок 1  & 4\,2\,0\,3\,1\\
Потомок 2  & 3\,4\,1\,0\,2\\
Мутация    & не выполнена.\\
\end{longtable}

\paragraph{Пара 2 (индексы $[3,2]$, точки $1,3$).}
\begin{longtable}{@{}p{0.18\linewidth}p{0.75\linewidth}@{}}
Родитель 1 & 0 \,|\, 2\,1\,3 \,|\, 4\\
Родитель 2 & 1 \,|\, 2\,0\,3 \,|\, 4\\
Потомок 1  & 1\,2\,0\,3\,4\\
Потомок 2  & 0\,2\,1\,3\,4\\
Мутация    & не выполнена.\\
\end{longtable}

\begin{center}
\begin{tabular}{>{\ttfamily}ccl}
\toprule
ID & Маршрут & Длина\\\midrule
P\_1 & 4\,2\,1\,0\,3 & 26\\
P\_2 & 1\,2\,0\,3\,4 & 28\\
P\_3 & 0\,2\,1\,3\,4 & 28\\
P\_4 & 1\,2\,0\,3\,4 & 28\\ \midrule
C\_9  & 4\,2\,0\,3\,1 & 28\\
C\_{10} & 3\,4\,1\,0\,2 & 32\\
C\_{11} & 1\,2\,0\,3\,4 & 28\\
C\_{12} & 0\,2\,1\,3\,4 & 28\\
\bottomrule
\end{tabular}
\end{center}

\subsection*{4.3 Итоговая популяция (после отбора)}

\begin{center}
\begin{tabular}{ccl}
\toprule
№ & Маршрут & Длина\\\midrule
1 & 4\,2\,1\,0\,3 & 26\\
2 & 4\,2\,0\,3\,1 & 28\\
3 & 1\,2\,0\,3\,4 & 28\\
4 & 0\,2\,1\,3\,4 & 28\\
\bottomrule
\end{tabular}
\end{center}

% =========================================================================== %
\section*{5. Лучший найденный маршрут}

\begin{itemize}
  \item Перестановка (индексация с нуля):\\
        \code{[4, 2, 1, 0, 3]}
  \item В терминах городов $1\!\dots 5$ (добавлено $+1$):\\
        \textbf{5 → 3 → 2 → 1 → 4 → 5}.
  \item Лучшая длина тура за 3 итерации: \textbf{26}.
\end{itemize}

\section*{6. Программная реализация}

\begin{lstlisting}[language=Python]
import random
import numpy as np

# --------------------------- CONSTANTS ---------------------------------------
MUTATION_RATE   = 0.01
NUM_CITIES      = 5
POP_SIZE        = 4
NUM_GENERATIONS = 3

# Distance matrix between the 5 cities
DIST = np.array(
    [
        [0, 4, 6, 2, 9],
        [4, 0, 3, 2, 9],
        [6, 3, 0, 5, 9],
        [2, 2, 5, 0, 8],
        [9, 9, 9, 8, 0],
    ]
)

# --------------------------------------------------------------------------- #
def path_len(route: list[int]) -> int:
    """
    Compute total tour length (objective value).

    This sums the distances for each consecutive edge in the permutation and
    finally adds the edge that closes the tour (last city → first city).
    """
    return sum(DIST[route[i], route[(i + 1) % NUM_CITIES]]
               for i in range(NUM_CITIES))

# --------------------------------------------------------------------------- #
def init_population() -> list[list[int]]:
    """Create initial population of random permutations."""
    return [random.sample(range(NUM_CITIES), NUM_CITIES)
            for _ in range(POP_SIZE)]

# --------------------------------------------------------------------------- #
def mutate(route: list[int]) -> bool:
    """Swap-mutation: exchange two random positions (returns True if mutated)."""
    if random.random() < MUTATION_RATE:
        i, j = random.sample(range(NUM_CITIES), 2)
        route[i], route[j] = route[j], route[i]
        return True
    return False

# -------- Order-Crossover with fragment exchange and “left-fill” ------------ #
def ox(parent_host: list[int], parent_donor: list[int], a: int, b: int) -> list[int]:
    """
    1. Exchange step: copy slice parent_donor[a:b] into child
       (this is where offspring inherit the foreign fragment).
    2. Filling step: walk through parent_host (starting at a+1, cyclic)
       and insert missing cities into the LEFT-most free “stars”,
       then continue on the right side of the exchanged slice.

    • The fragment exchange happens once at line 'child[a:b+1] = ...'.
    • The subsequent loop fills remaining gaps preserving order
      and ensuring no duplicates.
    """
    n = len(parent_host)
    child = [-1] * n                              # empty child
    child[a:b+1] = parent_donor[a:b+1]            # *** EXCHANGE SEGMENT ***

    # insertion order: all indices left of slice, then right of slice
    positions = list(range(0, a)) + list(range(b + 1, n))

    idx_host = (a + 1) % n                        # start scanning host
    for _ in range(n):
        gene = parent_host[idx_host]
        if gene not in child:
            pos = positions.pop(0)               # *** FILL FIRST LEFT STAR ***
            child[pos] = gene
        idx_host = (idx_host + 1) % n
    return child

# ------------------------------ GA MAIN LOOP -------------------------------- #
def genetic_algorithm():
    population = init_population()

    for gen in range(NUM_GENERATIONS):
        fitness = np.array([path_len(r) for r in population])
        probs = 1 / fitness
        probs /= probs.sum() # better tour gets more chance to be the parent

        print(f"\nGeneration {gen + 1}:")
        print("Population:", population)
        print("Distances:", fitness)

        new_population = []
        pair_no = 0
        while len(new_population) < POP_SIZE:
            i1, i2 = np.random.choice(len(population), 2, replace=False, p=probs)
            p1, p2 = population[i1], population[i2]
            pair_no += 1

            # inner crossover points (no edge positions)
            a, b = sorted(random.sample(range(1, NUM_CITIES - 1), 2))

            print(f"\nPair {pair_no}: [{i1}, {i2}]  cuts {a},{b}")
            print("Parent 1:", *p1[:a], "|", *p1[a:b + 1], "|", *p1[b + 1:])
            print("Parent 2:", *p2[:a], "|", *p2[a:b + 1], "|", *p2[b + 1:])

            child1 = ox(p1, p2, a, b)
            child2 = ox(p2, p1, a, b)
            print("Child 1:", child1)
            print("Child 2:", child2)

            if mutate(child1):
                print("Child 1 MUTATED:", child1)
            if mutate(child2):
                print("Child 2 MUTATED:", child2)

            new_population.extend([child1, child2])

        # Elitism: join old + new individuals and keep the best 4 tours
        # (lowest objective value) to form the next generation.
        population = sorted(population + new_population, key=path_len)[:POP_SIZE]

        print("\nEnlarged population:", population)
        print("Distances:", [path_len(r) for r in population])

    best = min(population, key=path_len)
    return best, path_len(best)

# --------------------------------------------------------------------------- #
if __name__ == "__main__":
    best_route, best_dist = genetic_algorithm()
    print(f"\nBest route: {best_route}, Distance: {best_dist}")

\end{lstlisting}

\end{document}
