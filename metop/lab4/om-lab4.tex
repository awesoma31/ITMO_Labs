\documentclass{article}
\usepackage[utf8]{inputenc}
\usepackage[russian]{babel}
\usepackage{amsmath, amssymb}
\usepackage{listings}
\usepackage{color}
\usepackage{geometry}
\geometry{a4paper, left=25mm, right=25mm, top=20mm, bottom=20mm}

\lstset{
    basicstyle=\ttfamily\small,
    columns=fullflexible,
    breaklines=true,
    frame=single,
    captionpos=b,
}

\begin{document}

\title{Отчёт по поиску экстремума функции методами оптимизации}
\author{ }
\date{\today}
\maketitle

\section{Постановка задачи}
Найти минимум функции
\[
f(x_1,x_2)=2x_1^2+4x_2^2-5x_1x_2+11x_1+8x_2-3
\]
методами:
\begin{itemize}
    \item покоординатного спуска,
    \item градиентного спуска с фиксированным шагом,
    \item наискорейшего спуска (с поиском оптимального шага).
\end{itemize}
Точность остановки алгоритмов: \(\varepsilon=10^{-4}\). Во всех случаях начальное приближение выбрано как
\[
x^{(0)}=(0,0).
\]

\section{Метод покоординатного спуска}
\subsection{Алгоритм}
При покоординатном спуске функция минимизируется поочерёдно по каждому аргументу, то есть:
\begin{enumerate}
    \item зафиксировав \(x_2\) (и прочие координаты), минимизируем функцию по \(x_1\);
    \item затем, зафиксировав уже обновлённое \(x_1\), минимизируем функцию по \(x_2\).
\end{enumerate}
На каждой подзадаче используется решение одномерной задачи оптимизации (приравнивание производной к нулю).

\subsection{Ручные вычисления}
\textbf{Итерация 1:} \\
Начальное приближение: 
\[
M^0=(x_1^0,x_2^0)=(0,0).
\]

\medskip
\emph{Шаг 1. Минимизация по \(x_1\) при фиксированном \(x_2=0\).} \\
При \(x_2=0\) функция принимает вид:
\[
f(x_1,0)=2x_1^2+11x_1-3.
\]
Найдём стационарную точку, приравняв производную по \(x_1\) к нулю:
\[
\frac{d f}{dx_1}=4x_1+11=0 \quad \Longrightarrow \quad x_1^*=-\frac{11}{4}=-2.75.
\]
Получаем обновлённую точку:
\[
M^1=(-2.75,\,0).
\]

\medskip
\emph{Шаг 2. Минимизация по \(x_2\) при фиксированном \(x_1=-2.75\).} \\
Подставляем \(x_1=-2.75\) в \(f(x_1,x_2)\):
\[
f(-2.75,x_2)=2(2.75^2)+4x_2^2-5(-2.75)x_2+11(-2.75)+8x_2-3.
\]
Вычислим части по шагам:
\[
2(2.75^2)=2\cdot7.5625=15.125;
\]
\[
-5(-2.75)x_2=+13.75x_2,\quad\text{и }\,11(-2.75)=-30.25;
\]
\[
8x_2\quad \text{и } -3.
\]
Таким образом, соберём:
\[
f(-2.75,x_2)=4x_2^2+(13.75+8)x_2+(15.125-30.25-3)=4x_2^2+21.75x_2-18.125.
\]
Приравнивая производную по \(x_2\) к нулю:
\[
\frac{d f}{dx_2}=8x_2+21.75=0\quad\Longrightarrow\quad x_2^*=-\frac{21.75}{8}=-2.71875.
\]
Получаем:
\[
M^2=(-2.75,\,-2.71875).
\]

\medskip
\textbf{Итерация 2:} \\
Начинаем с \(M^2=(-2.75,\,-2.71875)\).

\medskip
\emph{Шаг 1. Минимизация по \(x_1\) при фиксированном \(x_2=-2.71875\).} \\
Запишем функцию:
\[
f(x_1,-2.71875)=2x_1^2+4(2.71875^2)-5x_1(-2.71875)+11x_1+8(-2.71875)-3.
\]
Заметим, что:
\[
4(2.71875^2)=29.5625,\quad -5x_1(-2.71875)=+13.59375\,x_1,\quad 11x_1\,\text{даёт суммарно }24.59375\,x_1,
\]
а константная часть:
\[
29.5625+8(-2.71875)-3=29.5625-21.75-3=4.8125.
\]
Таким образом,
\[
f(x_1,-2.71875)=2x_1^2+24.59375\,x_1+4.8125.
\]
Приравниваем производную:
\[
\frac{d f}{dx_1}=4x_1+24.59375=0\quad\Longrightarrow\quad x_1^*=-\frac{24.59375}{4}=-6.14844.
\]
Получаем:
\[
M^3=(-6.14844,\,-2.71875).
\]

\medskip
\emph{Шаг 2. Минимизация по \(x_2\) при фиксированном \(x_1=-6.14844\).} \\
При \(x_1=-6.14844\) функция:
\[
f(-6.14844,x_2)=4x_2^2+\alpha\, x_2+\beta.
\]
Вычисления дают:
\[
\alpha\approx38.74219,\quad \beta\approx4.97290.
\]
Приравниваем производную:
\[
8x_2+38.74219=0\quad\Longrightarrow\quad x_2^*=-\frac{38.74219}{8}=-4.84277.
\]
Получаем:
\[
M^4=(-6.14844,\,-4.84277).
\]

\medskip
\textbf{Итерация 3:} \\
Начинаем с \(M^4=(-6.14844,\,-4.84277)\).

\medskip
\emph{Шаг 1. Минимизация по \(x_1\).} \\
При вычислении аналогичным способом получаем уравнение:
\[
4x_1+35.21387=0\quad\Longrightarrow\quad x_1^*=-\frac{35.21387}{4}=-8.80347.
\]
Обновлённая точка:
\[
M^5=(-8.80347,\,-4.84277).
\]

\medskip
\emph{Шаг 2. Минимизация по \(x_2\).} \\
При фиксированном \(x_1=-8.80347\) получаем:
\[
8x_2+52.01735=0\quad\Longrightarrow\quad x_2^*=-\frac{52.01735}{8}=-6.50217.
\]
Таким образом,
\[
M^6=(-8.80347,\,-6.50217).
\]

\medskip
\textbf{Вывод для покоординатного спуска:}\\
После трёх полных циклов (каждый цикл --- оптимизация по \(x_1\) и \(x_2\)) получаем приближённое решение:
\[
x^*\approx (-8.80347,\,-6.50217).
\]

\section{Метод градиентного спуска с фиксированным шагом}
\subsection{Алгоритм}
Итерационное правило имеет вид:
\[
x^{(k+1)}=x^{(k)}-\lambda\,\nabla f(x^{(k)}),
\]
где фиксированный шаг выбран, как в данном примере, \(\lambda=0.1\). Градиент функции:
\[
\nabla f(x_1,x_2)=
\begin{pmatrix}
4x_1-5x_2+11\\[1mm]
8x_2-5x_1+8
\end{pmatrix}.
\]

\subsection{Ручные вычисления}
\textbf{Итерация 1:}\\
При \(x^{(0)}=(0,0)\):
\[
\nabla f(0,0)=
\begin{pmatrix}
11\\
8
\end{pmatrix}.
\]
Обновляем:
\[
x^{(1)}=(0,0)-0.1\,(11,8)=(-1.1,-0.8).
\]
Вычисление функции даёт:
\[
f(-1.1,-0.8)\approx-20.92.
\]

\medskip
\textbf{Итерация 2:}\\
При \(x^{(1)}=(-1.1,-0.8)\):
\[
\nabla f(-1.1,-0.8)=
\begin{pmatrix}
4(-1.1)-5(-0.8)+11\\[1mm]
8(-0.8)-5(-1.1)+8
\end{pmatrix}
=\begin{pmatrix}
10.6\\[1mm]
7.1
\end{pmatrix}.
\]
Обновляем:
\[
x^{(2)}=(-1.1,-0.8)-0.1\,(10.6,7.1)=(-2.16,-1.51),
\]
а значение функции:
\[
f(-2.16,-1.51)\approx-36.70.
\]

\medskip
\textbf{Итерация 3:}\\
При \(x^{(2)}=(-2.16,-1.51)\):
\[
\nabla f(-2.16,-1.51)\approx
\begin{pmatrix}
9.91\\[1mm]
6.72
\end{pmatrix}.
\]
Обновляем:
\[
x^{(3)}=(-2.16,-1.51)-0.1\,(9.91,6.72)\approx(-3.151,-2.182),
\]
и функция равна:
\[
f(-3.151,-2.182)\approx-50.59.
\]

\medskip
\textbf{Вывод для градиентного спуска:}\\
После трёх итераций получаем:
\[
x^*\approx(-3.151,-2.182).
\]

\section{Метод наискорейшего спуска}
\subsection{Алгоритм}
Метод наискорейшего спуска отличается тем, что на каждом шаге выбирается оптимальный шаг \(\lambda_k\) по направлению антиградиента:
\[
x^{(k+1)}=x^{(k)}-\lambda_k\,\nabla f(x^{(k)}),
\]
где \(\lambda_k\) определяется как решение одномерной задачи:
\[
\min_{\lambda}\, f\Bigl(x^{(k)}-\lambda\,\nabla f(x^{(k)})\Bigr).
\]

\subsection{Ручные вычисления}
\textbf{Итерация 1:}\\
При \(x^{(0)}=(0,0)\) имеем:
\[
\nabla f(0,0)=
\begin{pmatrix}
11\\
8
\end{pmatrix}.
\]
Положим
\[
\varphi(\lambda)=f(-11\lambda,-8\lambda).
\]
Подставляем в функцию:
\begin{align*}
\varphi(\lambda)&=2(11\lambda)^2+4(8\lambda)^2-5\,(11\lambda)(8\lambda)+11(-11\lambda)+8(-8\lambda)-3\\[1mm]
&=242\lambda^2+256\lambda^2-440\lambda^2-121\lambda-64\lambda-3\\[1mm]
&=58\lambda^2-185\lambda-3.
\end{align*}
Найдем минимум \(\varphi(\lambda)\):
\[
\varphi'(\lambda)=116\lambda-185=0\quad\Longrightarrow\quad \lambda_0=\frac{185}{116}\approx1.59483.
\]
Обновляем:
\[
x^{(1)}=(0,0)-1.59483\,(11,8)\approx(-17.543,-12.7586).
\]
Вычисляем:
\[
f(x^{(1)})\approx-149.04.
\]

\medskip
\textbf{Итерация 2:}\\
При \(x^{(1)}=(-17.543,-12.7586)\):
\[
\nabla f(x^{(1)})=\begin{pmatrix}
4(-17.543)-5(-12.7586)+11\\[1mm]
8(-12.7586)-5(-17.543)+8
\end{pmatrix}\approx
\begin{pmatrix}
4.621\\[1mm]
-6.354
\end{pmatrix}.
\]
Для оптимального шага решается одномерная задача для функции
\[
\varphi(\lambda)=f\Bigl(x^{(1)}-\lambda\,\nabla f(x^{(1)})\Bigr),
\]
что после свёртки приводит к квадратному многочлену:
\[
\varphi(\lambda)=351.208\lambda^2-62.427\lambda-149.202.
\]
Из условия \(\varphi'(\lambda)=0\) получаем:
\[
702.416\lambda-62.427=0\quad\Longrightarrow\quad \lambda_1\approx0.0889.
\]
Обновляем:
\[
x^{(2)}=(-17.543,-12.7586)-0.0889\,(4.621,-6.354)\approx(-17.953,-12.1946),
\]
с \(f(x^{(2)})\approx-153.25\).

\medskip
\textbf{Итерация 3:}\\
При \(x^{(2)}=(-17.953,-12.1946)\) градиент:
\[
\nabla f(x^{(2)})\approx\begin{pmatrix}0.161\\0.208\end{pmatrix}
\]
почти равен нулю, поэтому дальнейшее обновление будет незначительным. Можно принять:
\[
x^{(3)}\approx x^{(2)}.
\]

\medskip
\textbf{Вывод для метода наискорейшего спуска:}\\
Приближённое решение после двух значимых итераций:
\[
x^*\approx(-17.953,-12.1946),
\]
что близко к аналитически найденному минимуму (решением \(\nabla f(x)=0\)):
\[
x^*=\left(-\frac{128}{7},\,-\frac{87}{7}\right)\approx(-18.2857,-12.4286).
\]

\section{Программная реализация на Python}
Ниже приведён пример программы, реализующей все описанные методы.

\begin{lstlisting}
import numpy as np
from scipy.optimize import minimize_scalar

def f(x):
    """Целевая функция: f(x1, x2) = 2x1^2 + 4x2^2 - 5x1x2 + 11x1 + 8x2 - 3"""
    return 2 * x[0]**2 + 4 * x[1]**2 - 5 * x[0] * x[1] + 11 * x[0] + 8 * x[1] - 3

def grad_f(x):
    """Градиент целевой функции"""
    return np.array([4 * x[0] - 5 * x[1] + 11, 8 * x[1] - 5 * x[0] + 8])

def coord_descent(x0, epsilon=1e-4, max_iter=1000):
    """
    Метод покоординатного спуска для минимизации функции f.

    Args:
        x0 (numpy.ndarray): Начальная точка.
        epsilon (float): Точность.
        max_iter (int): Максимальное количество итераций.

    Returns:
        tuple: Оптимальная точка и значение функции в ней.
    """

    x = x0.copy()
    f_values = [f(x)]
    x_history = [x.copy()]  # Store the history of x

    for _ in range(max_iter):
        x_prev = x.copy()

        # Минимизация по x1
        res_x1 = minimize_scalar(lambda x1: f([x1, x[1]]))
        x[0] = res_x1.x

        # Минимизация по x2
        res_x2 = minimize_scalar(lambda x2: f([x[0], x2]))
        x[1] = res_x2.x

        x_history.append(x.copy())
        f_values.append(f(x))

        if np.abs(f(x) - f(x_prev)) < epsilon:
            break

    return x, f(x), x_history, f_values


def gradient_descent(x0, epsilon=1e-4, learning_rate=0.01, max_iter=1000):
    """
    Метод градиентного спуска для минимизации функции f.

    Args:
        x0 (numpy.ndarray): Начальная точка.
        epsilon (float): Точность.
        learning_rate (float): Шаг градиентного спуска.
        max_iter (int): Максимальное количество итераций.

    Returns:
        tuple: Оптимальная точка и значение функции в ней.
    """

    x = x0.copy()
    f_values = [f(x)]
    x_history = [x.copy()]  # Store the history of x
    
    for _ in range(max_iter):
        x_prev = x.copy()
        grad = grad_f(x)
        x = x - learning_rate * grad

        x_history.append(x.copy())
        f_values.append(f(x))

        if np.linalg.norm(x - x_prev) < epsilon or np.abs(f(x) - f(x_prev)) < epsilon:
            break

    return x, f(x), x_history, f_values

def steepest_descent(x0, epsilon=1e-4, max_iter=1000):
    """
    Метод наискорейшего спуска для минимизации функции f.

    Args:
        x0 (numpy.ndarray): Начальная точка.
        epsilon (float): Точность.
        max_iter (int): Максимальное количество итераций.

    Returns:
        tuple: Оптимальная точка и значение функции в ней.
    """

    x = x0.copy()
    f_values = [f(x)]
    x_history = [x.copy()]

    for _ in range(max_iter):
        x_prev = x.copy()
        grad = grad_f(x)
        direction = -grad

        # Одномерная минимизация для нахождения оптимального шага
        res_alpha = minimize_scalar(lambda alpha: f(x + alpha * direction))
        alpha = res_alpha.x

        x = x + alpha * direction
        
        x_history.append(x.copy())
        f_values.append(f(x))


        if np.linalg.norm(x - x_prev) < epsilon or np.abs(f(x) - f(x_prev)) < epsilon:
            break

    return x, f(x), x_history, f_values

\end{lstlisting}

\end{document}
